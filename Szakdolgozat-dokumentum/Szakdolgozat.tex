\documentclass[centeredchapter]{thesis-ekf}
\usepackage[T1]{fontenc}
\usepackage[utf8]{inputenc}
\PassOptionsToPackage{defaults=hu-min}{magyar.ldf}
\usepackage[magyar]{babel}
\usepackage{graphicx,amsmath,amssymb,amsthm}
\graphicspath{{./images/}}
\footnotestyle{rule=fourth}

\newtheorem{tetel}{Tétel}[chapter]
\newtheorem{lemma}[tetel]{Lemma}
\theoremstyle{definition}
\newtheorem{definicio}[tetel]{Definíció}
\newtheorem{feladat}[tetel]{Feladat}
\theoremstyle{remark}
\newtheorem{megjegyzes}[tetel]{Megjegyzés}
\newtheorem*{megoldas}{Megoldás}

\logo{\includegraphics[width=9cm]{ekf-logo3}}
\institute{Matematikai és Informatikai Intézet}
\title{Szoftverfejlesztés Java EE platformon}
\authorcaption{Készítette:}
\author{Kosárkó Ákos \\ Programtervező informatikus Bsc szak}
\supervisorcaption{Témavezető:}
\supervisor{Tajti Tibor\\ tanársegéd}
\city{Eger}
\date{2016}

\begin{document}
\maketitle
\tableofcontents

\chapter*{Bevezetés}

2015 elején foglalkoztatni kezdett, hogy munkába álljak és pénzt keressek. Ezen célért az interneten böngészve rátaláltam egy iskolaszövetkezet hirdetésére, amely biztos munkát ígért az általuk kínált képzés elvégzése után. A képzés tárgyaként a JAVA EE volt megjelölve, amiről első körben nem tudtam, hogy micsoda, annyit sejtettem, hogy a Java programozási nyelvhez kapcsolódhat. Második nekifutásra utánaolvastam az interneten, az jött le, hogy ez a Java programozási nyelv rengeteg külső osztálykönyvtárral kibővített "változata". Ma már tudom, hogy ez ennél jóval több. Visszatérve az iskolaszövetkezet hirdetésére, kapva az alkalmon, jelentkeztem a hirdetésükre, és a felvételi interjú után alkalmasnak találtak arra, hogy részt vegyek a képzésükön. 

Így kerültem kapcsolatba a Java Enterprise Edition technológiával, amely, mint később világossá vált számomra, az iparban széles körben elterjedt és számos helyen használt eszköz. A szakdolgozat elkészítésével és az ehhez írt alkalmazás elkészítésével célom a technológia általam már ismert részének bemutatása, a technológia által nyújtott és számomra új megoldások megtalálása, és a gyakorlás. Mindezeket egy fiktív telekommunikációs cég számára készített alkalmazás, amellyel online ügyintézést tesznek lehetővé a régi és új ügyfeleik számára.

\chapter{Specifikáció}

\section{Az alkalmazás célja}

Az alkalmazás célja lehetővé tenni a szolgáltató ügyfelei számára, hogy online intézhessék az ügyeiket. Ezen ügyintézésbe a cég szolgáltatásaira való előfizetés, szolgáltatáslemondás, számlabefizetés és hibabejelentés tartozik, emellett a szolgáltató az oldalán az új akciókról és a szolgáltatással kapcsolatos hírekről tájékoztathatja a leendő és már meglévő ügyfeleit. 

\section{Képernyők}

Az alkalmazással végezhető műveletek természetesen láthatóak kell legyenek a használóik számára, erre a képernyők szolgálnak. Az alkalmazás webes böngészőkön keresztül érhető el a felhasználók számára, a képernyők is abban fognak megjelenni nekik.

\subsection{Főoldal}

\subsection{Bejelentkezési felület}

\subsection{Bemutatkozó oldal}

\subsection{Szolgáltatások}

\subsection{Akciók}

\subsection{Kapcsolat}

\subsection{Előfizetési oldal}

\subsection{Számlabefizetés}

\subsection{Hibabejelentés}

\section{A szoftver elkészítésére használt, és az azt támgató eszközök}

\subsection{Java Enterpreise Edition}

\subsection{WildFly Application Server}

\subsection{IntelliJ IDEA}

\subsection{Vaadin Framework}

\subsection{Git}

\chapter{A Java Enterprise Edition}

\section{Szakasz címe}

A Java EE API-k egy halmaza.

\subsection{Alszakasz címe}

\chapter{Az alkalmazás}

\section{Az adatbázismodell}

\subsection{Alszakasz címe}

\chapter{Összegzés}

\section{Szakasz címe}

\subsection{Alszakasz címe}


\begin{thebibliography}{1}
\bibitem{cimke} \textsc{Szerző}: Cím, Kiadó, Hely, évszám.
\end{thebibliography}
\end{document}
